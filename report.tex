% Report cours C++
% Artem Oboturov, Ludwig Brummer
% le 12 Janvrier 2013
\documentclass[a4paper]{article}
\usepackage[francais]{babel}
%\usepackage[framed]{mcode}
\usepackage[final,pdftex]{graphicx}
\usepackage{todonotes}
\usepackage[dvips]{epsfig}
\usepackage{listings}

\newtheorem{algo}{Algorithme}

\begin{document}
\lstset{language=C++,tabsize=2, numbers=left}
%\listoftodos

\title{Le projet Mosaic}
\author{
  Artem Oboturov\thanks{oboturov@telecom-paristech.fr}
  \\Ludwig Brummer\thanks{ludwig.brummer@mytum.de}
}
\date{le 21 Janvrier 2013}
\maketitle
\section{Description du projet}
Dans le cadre du Master 2 Mod\'elistaion Al\'eatoire \`a l'Universit\'e Paris VII Denis Diderot les \'etudiants  doivent r\'ealiser un projet de C++ en bin\^ome pour montrer leur capacit\'e de programmation.
Cette ann\'ee le cible du projet \'etait la r\'ealisation d'un photomosa\"ique, c'est-\`a-dire un programme qui reconstruit un image en rempla\c{c}ant des parties de cet image par des autres images.
Ce texte d\'ecrit le programme d'Artem Oboturov et Ludwig Brummer.

Dans ce qui suit on parlera de l'organisation du projet dans le chapitre 2 et donnera un mode d'emploi  dans la section 3. Puis la structure de code est \'expliqu\'ee dans la partie 4 et quelques algorithmes sont discut\'es dans le chapitre 5.
Enfin on montrera quelques r\'esultats. 

\section{Organisation de travail sur projet}
Un probl\`eme, que on a rencontr\'e en d\'ebut de travail sur le projet, c'\'etait la nature distributive de l'\'equipe.
Pour travailler effectivement \`a distance, on a trouv\'e un site specialis\'e en gestion de projets en ligne.
Le projet a \'et\'e organis\'e atour de site Github - le portal de travail collaborative sur les logiciels.
Ce site garde le code source et donne acc\'es vers ce code par syst\`eme de contr\^ole de versions Git. Par ailleurs il montre tous les contributions de chaque collaborateur ligne par ligne dans le code et donne des diagrammes d'activit\'e et productivi\'e des collaborateurs.\\
Il y a un composant de gestion de projet permettant \`a cr\'eer des billets sur le taches de d\'eveloppement.
De plus il est possible de faire des branches dans le projet pour travailler parallèlement sur des m\^emes fiches et les r\'eunifier apr\`es.\\
On a cr\'e\'e un projet priv\'e (non accessible par personnes non autoris\'ees).
\section{Instruction d'emploi}
Dans le chapitre suivant le mode d'emploi du programme est pr\'esent\'e.
\subsection{La biblioth\`eque des carreaux}
D'abord on a besoin d'une certaine quantit\'e des images de lesquelles la programme construira la mosa\"{i}que.
 Pour un bon r\'esultat on a besoin de cinquante images au moins dans le dossier "imageLibrary" dans le dossier du programme.

Tous ces images doivent avoir la m\^eme taille.
Pour y arriver, on peut utiliser le programme "library\_generator.exe" ou d'abord le commande "make library\_generator" s'il ne se trouve pas d\'ej\`a dans le dossier du programme.
 Le programme copiera tous les images dans le dossier "inputImages" dans le dans dossier "libraryImages" et r\'eduira les tailles des images \`a un taille quadratique qui peut \^etre sp\'ecifi\'e dans la ligne de commande.
 Agrandir des images n'est pas possible avec ce programme.
 
 Le programme peut \^etre \'execut\'e avec deux different options dans la ligne de commande.
 
 D'abord on peut sp\'ecefier la taille des images souhait\'e apr\`es la commande "{-}{-}size" avec un nombre entier positif.
 La taille d\'efaute est 100 pixels.
Alors un \'execution exemplaire serait ".\textbackslash library\_generator {-}{-}size 50".

Et puis il y a l'option "{-}{-}help" qui r\'ep\'etera toute l'information ces instructions.

Il faut faire attention: le programme ne va pas couper des parties des images mais seulement r\'eduire la taille alors les proportions peuvent \^etre chang\'e.
Ce fa\c{c}on de r\'eduire a quelque avantages et disavantages mais pourrait facilement \^etre changer par couper la partie d'image qui est "trop" et r\'eduire apr\`es.

\subsection{Le programme principal}
Maintenant que nous avons une biblioth\`eque assez grand et de la m\^eme taille dans le dossier "libraryImages", on doit effacer tous les images dans le dossier "inputImages" sauf ceux qui sont destin\'e \`a devenir des mosa\"iques.
Puis on peut ex\'ecuter la programme ".\textbackslash m2mo\_brummer\_oboturov\_project.exe".
L\`a aussi il y a des options qui peuvent \^etre ajout\'e dans la ligne de commande.

"{-}{-}help" ou "-h" ouvrira plus ou moins les m\^emes informations que ce paragraphe mais n'ex\'ecutera pas le mosaic maker.

Avec les options "{-}{-}mse" ou "-a", "{-}{-}meancolor" ou "-b" et "{-}{-}mcmse" ou "-c" on peut choisir entre des mesures de divergence.

Avec l'option "{-}{-}tilesize" et un nombre positif entier on peut chosir la taille des carreaux, c'est-\`a-dire les sections qui sont remplacer par des autres images.

Les options peuvent \^etre \'ecrit dans n'importe quel ordre.
Les d\'efautes sont la mesure de Monte Carlo et un taille de 20 pixels.

Le programme annoncera quand un mosa\"ique est fini.
 Ils sont sauvegard\'es dans le dossier "outputImages". 

\section{Le progiciel et son code source}
De mani\`ere g\'en\'erale, la documentation pour chaque classe est r\'ealis\'e en forme de commentaires des m\'ethodes.

Dans notre travail on a largement r\'eutilis\'e le projet fournit par Christian Konrad.

\subsection{R\'ealisation des Images}
Cette classe d\'efinit le concept d'Image.

\lstset{caption={La classe Image},label={cpp-class-image}}
\begin{lstlisting}
class Image
{
public: 
	typedef BlockIterator iterator;
	
	// Defined in the project provided by C. Konrad
	Image(string filename);
	Image(img_size_t width, img_size_t height);
	Image(Image &img);
	~Image();
	
	// Image dimension accessor methods.
	img_size_t get_width() const;
	img_size_t get_height() const;
	
	// Provide iterator over Blocks which is the Image is divided into.
	iterator begin(const img_size_t height, const img_size_t width);
	iterator end(const img_size_t height, const img_size_t width);
	
	// Persist image from memory into a file.
	void save(string filename);
	
	// Method to flip Image horizontally.
	void flip_horizontally();
	
	// Accessor to a pixel at specified coordinate.
	img_color_t& operator()
		(img_coord_t x, img_coord_t y, img_color_layer_t layer) const;
	
	// Scale this Image into an Image of different size (which MUST be defined
	// in advance for a new Image).
	Image& scale_to(Image& downscaled) const; 
};
\end{lstlisting}

\subsection{D\'ecoupage des Images en Blocs}
Chaque Image peut-\^etre repr\'esent\'ee comme une s\'equence des blocs.
Bloc est un rectangle correspondant \`a une partie de l'Image et est d\'efini par ces coordonn\'ees de gauche-haute et de droit-basse.
C'est un concept naturel pour d\'efinir des algorithmes sur les parties matricielles des Images.
Une nuance de r\'ealisation: le bloc ne alloue pas de m\'empire pour stocker des pixels d'Image auxquelles il correspond.

\lstset{caption={La classe Block},label={cpp-class-block}}
\begin{lstlisting}
class Block
{
public:
	Block(const Image* image, const img_coord_t top, const img_coord_t left,
		const img_size_t height, const img_size_t width);
		
	// Accessors to Block dimensions.
	img_coord_t get_left() const { return left; }
	img_coord_t get_top() const { return top; }
	img_size_t get_height() const { return height; }
	img_size_t get_width() const { return width; }
	
	// Get pointer to Image this Block references to.
	const Image* get_img() const { return img; }
	
	// Copy content from another Block into this Block.
	void copy_content(Block& to_copy);
};
\end{lstlisting}

On peux parcourir des Blocs d'Image avec un iterator BlockIterator.
On a conceptualis\'e, que l'utilisateur de la classe Image va la parcourir de fa\c{c}on d\'efini par le Forward-Iterator.

\lstset{caption={La classe BlockIterator},label={cpp-class-block-iterator}}
\begin{lstlisting}
class BlockIterator
{
public:
	BlockIterator(Image* image, const img_size_t block_height,
		const img_size_t block_width);

	// Reference to an Instance representing the End of a BlockIterator.	
	BlockIterator& end();

	// Standard methods for a Forward-Iterator.
	Block& operator*();
	Block* operator->();
	BlockIterator& operator++();
	bool operator==(const BlockIterator&) const;
	bool operator!=(const BlockIterator&) const;
};
\end{lstlisting}

\subsection{Cr\'eation de mosa\"ique}
La mesure de divergence est une classe polymorphe.
Donc, chaque mesure est d\'efini dans une classe concret, qui r\'ealise une distance de divergence entre deux blocs.

\lstset{caption={La mesure de divergence},label={cpp-class-divergence-measure}}
\begin{lstlisting}
class DivergenceMeasure
{
public:
	// Compare blocks with the Mean Square Error proximity measure
	// and returns a value in [0,1].
	virtual float compute(const Block& lhs, const Block& rhs) const = 0;
	virtual ~DivergenceMeasure() = 0;
};
\end{lstlisting}

Pour augmenter la performance de parcours de la biblioth\`eque des images, on a fait un cache de ces images en fixant leur taille \`a BLOCK\_SZ.

\lstset{caption={L'algorothme de cr\'eation de mosa\"ique},label={cpp-mosaic}}
\begin{lstlisting}
ImageLibrary& archive;
Image* target_image;
DivergenceMeasure* divergence;
int BLOCK_SZ;

Image::iterator begin = target_image->begin(BLOCK_SZ, BLOCK_SZ),
	end = target_image->end(BLOCK_SZ, BLOCK_SZ);
// Scale down images from library.
Image	library(archive.size() * BLOCK_SZ, BLOCK_SZ),
	downscaled_img(BLOCK_SZ, BLOCK_SZ);
size_t cnt = 0;
for (ImageLibrary::storage::const_iterator image_in_lib_it = archive.begin();
	image_in_lib_it != archive.end(); ++image_in_lib_it, ++cnt)
{
	(*image_in_lib_it)->scale_to(downscaled_img);
	Block downscaled_block(&downscaled_img, 0, 0, BLOCK_SZ, BLOCK_SZ);
	Block(&library, 0, cnt * BLOCK_SZ, BLOCK_SZ, BLOCK_SZ)
		.copy_content(downscaled_block);
}
// Create mosaic.
for (Image::iterator blockIt = begin; blockIt != end; ++blockIt)
{
	float prox = 0.f;
	Image::iterator	begin = library.begin(BLOCK_SZ, BLOCK_SZ),
			end = library.end(BLOCK_SZ, BLOCK_SZ);
	Block chosen_tile(NULL, 0, 0, BLOCK_SZ, BLOCK_SZ);
	for (Image::iterator libIt = begin; libIt != end; ++libIt)
	{
		float div_res = (*divergence).compute(*libIt, *blockIt);
		if (div_res > prox)
		{
			prox = div_res;
			chosen_tile = *libIt;
		}
	}
	(*blockIt).copy_content(chosen_tile);
	cout << "*";
	cout.flush();
}
\end{lstlisting}

\subsection{Ex\'ecution avec des options de ligne des commandes}
Le progiciel soutien les param\`etres de ligne des commandes.
Cette fonctionnalit\'e est r\'esalis\'e avec la biblioth\`eque 'Getopt' de stdlibc.

\subsection{Le library\_generator}
On a cr\'e\'e une utilit\'e pour convertir les images arbitraires en format JPEG vers la taille de $ 100px \times 100px $.
Pour la compiler il faut appeler `make library\_generator`.
Les instructions d'utilisations sont accessible, si quelqu'un l'appelle avec un param\`etre '-h'.

\section{Algorithmique}

\subsection{L'algorithme de sous\'echantillonnage lin\'eaire}

L'algorithme de sous\'echantillonnage lin\'eaire ou de scaling est utilis\'e pour changer tout les images destin\'e pour la bibliothèque \`a une taille uniforme.
Ici on doit se rendre compte qu'il est seulement capable de diminuer la taille des images car c'est l'usage r\'ealiste pendant un construction d'un mosa\"{i}que.  
Pendant un agrandissement il y aura des fautes.

L'algorithme se trouve dans la fonction Image\& Image::scale\_to(Image\& downscaled) dans les fichiers image.h et image.cpp.
Alors il fait partie de la classe Image comme on peut voir dans le code.
La fonction re\c{c}oit comme param\'etre la reference d'une image "downscaled" qui a la taille \`a laquelle on veut changer l'image originale.

On peut trouver le code dans le Listing \ref{algo-scaling}.
\lstset{caption={L'algorithme de sous\'echantillonnage lin\'eaire},label={algo-scaling}}
\begin{lstlisting}
img_size_t	desiredw = downscaled.get_width(),
			desiredh = downscaled.get_height();

float paceh = height/desiredh;
float pacew = width/desiredw;
float norm = paceh*pacew;
int y = 0;// pixel positions for big image
int x = 0;
for(int i = 0; i < desiredw; i++)
{
	x=i*pacew;
	for(int j = 0; j < desiredh; j++)
	{
	y = j*paceh;
		for(img_color_layer_t layer = 0; layer < LAYER_CNT; layer++)
		{
			float color = 0;
			for (int cnth = 0; cnth < int(paceh); cnth++)
			{
				for (int cntw = 0; cntw < int(pacew); cntw++)
					color += (float)(*this)(x + cntw, y + cnth, layer);
			}
			downscaled(i, j, layer) = color/norm;
		}
	}	
}
\end{lstlisting}

 Grosso modo l'algorithme prend un carr\'e des plusieurs pixels, calcule la moyenne des couleurs et met un pixel avec le couleur moyen \`a la place de ces pixels.
C'est illustr\'e par Figure \ref{fig:carres}.
Comme \c{c}a, la taille d'un image est r\'eduit.
La prise de la moyenne est fait s\'epar\'ement pour  chaque couleur (boucle l.15-24) en additionnant tout les intensit\'es de ce couleur dans pixels du carr\'e choisi (l. 21) et les divisant par la taille du carr\'e (norm dans le code, l. 23).

Cette approche de r\'eduction est plut\^ot \'evidente. La difficult\'e est de choisir les bonnes tailles pour les carr\'es \`a r\'eduire.
C'est r\'ealis\'e par les lignes 4-14 et illustr\'e par Figure \ref{fig:carres}.
Ici on utilise la fa\c{c}on comment C++ sauvegarde et transforme des variables.
Des pas, alors les longueurs des c\^otes du carr\'e \`a r\'eduire,  sont d\'efinis comme floats aux lignes 4 et 5 car la proportion entre la hauteur respectivement la largeur d'un image et la hauteur souhait\'e respectivement la largeur souhait\'e n'est pas toujours un entier.
Mais les carr\'es \`a r\'eduire ne peuvent avoir qu'un longueur entier.
Alors on d\'efinie la position \`a gauche et en haut   dan le carr\'e \`a r\'eduire comme couplet des entiers $(x,y)$ mais fait le calcul de la position avec des floats.
Comme \c{c}a, il n'y a pas des erreurs dans le calcul, mais les positions sont automatiquement transform\'e en entiers et les tailles des carr\'es vacillent entre les deux entiers qui entourent les pas.
\begin{figure}
  \centerline{\psfig{figure=scale_to.png,width=160mm}}
  \caption{Un illustration du choix des bonnes tailles pour les carr\'es \`a r\'eduire}
  \label{fig:carres}
\end{figure}

\subsection{Les mesures de divergence}
\subsubsection{La mesure de erreur des moindres carr\'es}

$X,Y,C$ correspondent aux tailles en dimension horizontal, vertical et couleur. $C$ est toujours 3. 
\begin{equation}
err=\frac{1}{X\cdot Y\cdot C}\sum_{x,y,c}\left(\frac{block(x,y,c)-tile_{image}(x,y,c)}{255}\right)^2
\end{equation}
\subsubsection{La mesure de couleurs moyens}

\subsubsection{La mesure de Monte Carlo}
Soient $(Z_{x,y,c})$ une suite des variables al\'eatoires i.i.d. de la loi Bernoulli avec $p~=~0,25$.
\begin{equation}
err=\frac{1}{\sum_{x,y,c}Z_{x,y,c}}\sum_{x,y,c}\left(\frac{block(x,y,c)-tile_{image}(x,y,c)}{255}\right)^2\cdot Z_{x,y,c}
\end{equation}



\section{Resultats}
Comparaison des mesures de divergence: vitesse, qualit\'e (subjectif)
Exp\'erimente sur le m\^eme ordinateur avec 200 (nombre aléatoire mais fixe) images entre les mesures.
 Prends le temps. Apr\`es montre les r\'esultats et discute un peu la qualit\'e. 

%\begin{figure}[t]
%	\begin{center}
%		\lstinputlisting{trace_frontieres.m}
%	\end{center}
%	\caption{Fichier {\em trace\_frontieres.m}}
%	\label{fig:OptimizationProblemWithAuthorizedSalesListing1}
%\end{figure}

\end{document}

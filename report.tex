% Report cours C++
% Artem Oboturov, Ludwig Brummer
% le 12 Janvrier 2013
\documentclass[a4paper]{article}

%\usepackage[framed]{mcode}
\usepackage[final,pdftex]{graphicx}
\usepackage{todonotes}

\begin{document}

\listoftodos

\title{Le projet Mosaic}
\author{
  Artem Oboturov\thanks{oboturov@telecom-paristech.fr}
  \\Ludwig Brummer\thanks{\todo[inline]{Ludwig, put your email here}}
}
\date{le 21 Janvrier 2013}
\maketitle

\section{Organisation de travail sur projet}
Un probl\`eme, que on a rencontr\'e en d\'ebut de travail sur le projet, c'\'etait la nature distributive de l'\'equipe.
Pour travailler effectivement \`a distance, on a trouv\'e un site specialis\'e en gestion de projets en ligne.
Le projet a \'et\'e organis\'e atour de site Github - le portal de travail collaborative sur les logiciels.
Ce site garde le code source et donne acc\'es vers ce code par syst\`eme de contr\^ole de versions Git.
Il y a un composant de gestion de projet permettant a cr\'eer des billets sur le taches de d\'eveloppement.

On a cr\'e\'e un projet priv\'e (non accessible par personnes non autoris\'ees).

\section{Structure de code}

\section{Algorithmique}

\section{Contributions au projet}

\section{Resultats}

%\begin{figure}[t]
%	\begin{center}
%		\lstinputlisting{trace_frontieres.m}
%	\end{center}
%	\caption{Fichier {\em trace\_frontieres.m}}
%	\label{fig:OptimizationProblemWithAuthorizedSalesListing1}
%\end{figure}

\end{document}
